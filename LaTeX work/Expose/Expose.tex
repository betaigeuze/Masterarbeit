\documentclass[a4paper, 12pt]{article}
% "couple of pages" about:
% Give an overview over the field of random forest visualization

% TaxonTree, SpaceTree DoubleTree TreeWiz => examples for scalable tree visualizations
% Woodburn, makes the argument for Treemaps, icicle charts and sunburst charts being most
% prevalent

% What has been done:
% Around that time there were approaches like RAFT by Breimann and ReFINE by the dutch author
% BaobabView is an example for a tree visualization approach, that is however not scalable
% iForest and ExMatrix are the best visualizations to date

% What I want to convey:
% 2 main problems:
% I think that using tree visualizations is not smart and instead use something more
% scalable.
% => Tree visualisations are scalable for LARGE trees, but not for MANY trees
% These approaches target professionals with a background in data science.
% If you want to approach true domain experts, the solution has to be MORE ACCESSIBLE


\usepackage[T1]{fontenc}
\usepackage{amsmath}
\usepackage{apacite}
\usepackage{enumitem}


\title{Expose - Explainability Machine Learning - Visualization of Random Forests}
\author{Fabio Rougier}
\date{\today}

\begin{document}

\maketitle

\tableofcontents
\clearpage
\section{Introduction}
Random Forests (RF) are a powerful ensemble method with a low barrier of entry. Because of their
ease of use and performance they are used in many applications. However, they fall short
when it comes to transparency.
RFs yield a multitude of valuable insights into their decision making and the data they
process. Visualizing this is usually a challenge because of the inherent scale that of a RF. 
In this work we want to look at the approaches trying to overcome this limitation.

\section{Visualizing Decision Trees}
An obvious approach to visualizing a RF is to inspect the decision trees, constituting the
RF. One example for the visualization of a decision tree is \textit{BaobabView}
\cite{van2011baobabview}. As many other approaches visualizing decision trees, it utilizes
Node-Link Diagrams (NLDs) as its' main visualization. A confusion matrix yields additional
insights into the relations of the underlying features.
However this visualization does fall short when trees grow too large, as it becomes hard to
inspect every individual node and branch of the tree.
This is one of the problems that \textit{TaxonTree} tries to overcome 
\cite{parr2003taxontree}. It uses a tree visualization approach that is scalable for large 
trees by adding the possibility to zoom, browse and search the tree. While this does deal
help if a tree grows too large, it does not provide any insight on the general structure of
the tree as a whole.
\linebreak
\linebreak
Generalizing either of the approaches towards RFs is also non trivial. Both simply lack the
scalability in the desired dimension. It also leads into a dangerous of focusing too much on
the structure of individual trees. RFs - as all ensemble methods - arrive at their decisions
by combining the decisions of the individual trees. Inspecting and even understanding
individual trees, will only yield limited insights over the RF.

\section{Visualizing Random Forests}
To fully understand the structure and decision making of a RF, many aspects of the RF have to
be conveyed by the visualization. First of all the typical indicators like a 
\textit{mean squared error} or a \textit{mean average error} of any machine Learning model
give an indication of the overall performance. Adiitionaly there are some metrics specific
to RFs that should be included to get a deeper understanding of the specific instance of the
RF, like the \textit{mean impurity} of the individual trees or the 
\textit{out of bag error}. With its' unique way of giving a distance measure for features,
a confusion matrix can also yield valuable insights to the relations of the features and how
the RF interprets them. As stated before, the visualization of RFs heavily relies on how it
overcomes the scalability issues of RFs.

\subsection{RAFT}
The RAFT visualization is one of the first of its' kind and features all of the
aforementioned criteria. 


\section{Summary}
\cite{zhao2018iforest}

\clearpage
\bibliographystyle{apacite}
\bibliography{ExposeReferences}
\end{document}