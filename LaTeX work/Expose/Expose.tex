\documentclass[a4paper, 12pt]{article}
% "couple of pages" about:
% Give an overview over the field of random forest visualization

% TaxonTree, SpaceTree DoubleTree TreeWiz => examples for scalable tree visualizations
% Woodburn, makes the argument for Treemaps, icicle charts and sunburst charts being most
% prevalent

% What has been done:
% Around that time there were approaches like RAFT by Breimann and ReFINE by the dutch author
% BaobabView is an example for a tree visualization approach, that is however not scalable
% iForest and ExMatrix are the best visualizations to date

% What I want to convey:
% 2 main problems:
% I think that using tree visualizations is not smart and instead use something more
% scalable.
% => Tree visualisations are scalable for LARGE trees, but not for MANY trees
% These approaches target professionals with a background in data science.
% If you want to approach true domain experts, the solution has to be MORE ACCESSIBLE


\usepackage[T1]{fontenc}
\usepackage{amsmath}
\usepackage{apacite}
\usepackage{enumitem}


\title{Expose - Explainability Machine Learning - Visualization of Random Forests}
\author{Fabio Rougier}
\date{\today}

\begin{document}

\maketitle

\tableofcontents
\clearpage
\section{Introduction}
Random Forests (RF) are a powerful ensemble method with a low barrier of entry. Because of their
ease of use and performance they are used in many applications. However, they fall short
when it comes to transparency.
RFs yield a multitude of valuable insights into their decision making and the data they
process. Visualizing this is usually a challenge because of the inherent scale that of a RF. 
In this work we want to look at the approaches trying to overcome this limitation.

\section{Visualizing Decision Trees}
An obvious approach to visualizing a RF is to inspect the decision trees, constituting the
RF. One example for the visualization of a decision tree is \textit{BaobabView}
\cite{van2011baobabview}. As many other approaches visualizing decision trees, it utilizes
Node-Link Diagrams (NLDs) as its' main visualization. A confusion matrix yields additional
insights into the relations of the underlying features.
However this visualization does fall short when trees grow too large, as it becomes hard to
inspect every individual node and branch of the tree.
This is one of the problems that \textit{TaxonTree} tries to overcome 
\cite{parr2003taxontree}. It uses a tree visualization approach that is scalable for large 
trees by adding the possibility to zoom, browse and search the tree. While this does deal
help if a tree grows too large, it does not provide any insight on the general structure of
the tree as a whole.
\linebreak
\linebreak
Generalizing either of the approaches towards RFs is also non trivial. Both simply lack the
scalability in the desired dimension. It also leads into a dangerous of focusing too much on
the structure of individual trees. RFs - as all ensemble methods - arrive at their decisions
by combining the decisions of the individual trees. Inspecting and even understanding
individual trees, will only yield limited insights over the RF.

\section{Visualizing Random Forests}
To fully understand the structure and decision making of a RF, many aspects of the RF have to
be conveyed by the visualization. First of all the typical indicators like a 
\textit{mean squared error} or a \textit{mean average error} of any machine Learning model
give an indication of the overall performance. Adiitionaly there are some metrics specific
to RFs that should be included to get a deeper understanding of the specific instance of the
RF, like the \textit{mean impurity} of the individual trees or the 
\textit{out of bag error}. With its' unique way of giving a distance measure for features,
a confusion matrix can also yield valuable insights to the relations of the features and how
the RF interprets them. As stated before, the visualization of RFs heavily relies on how it
overcomes the scalability issues of RFs.

\subsection{ReFine}


\subsection{iForest}
The \textit{iForest} visualization is one of the most promising visualization approaches for RFs.
\cite{zhao2018iforest} It focuses on the interpretability of the RF and raises the concern
that many domains currently would not even consider using RFs because of their lack thereof.
The authors also approach the understanding of RFs by coming from two different angles:
\textit{Feature Analysis} and \textit{Case Based Reasoning}.
Both require very different charts and give the user valuable insight into the dataset and
the behaviour of the RF. The elaborate, dashboard-like web application utilizes the benefits
of small multiples and has interconnected and interactive charts, each devoted to offer a
specific insight. One especially unique part of the dashboard is the use of
\textit{Partial Dependence Plots} to display the RF's classification behaviour in regard to
each feature. This is supported by a bar chart of the feature's distribution in order to
support this view with more context.
While the dashboard is a very powerful tool, it can be very overwhelming for users. It also
requires some in-depth knowledge about RFs in order to come to conclusions or a required
action instruction. This tool is clearly aimed at supporting data scientists to understand
their own creations and not suited for a domain expert.

\subsection{ExMatrix}
The \textit{ExMatrix} follows an out-of-the-box-thinking approach by breaking up machine
learning models into a set of rules \cite{neto2020explainable} \cite{ming2018rulematrix}. It is therefore very powerful because it is not limited
to be used with RFs only, but can be applied to almost any kind of machine learning model.
It is best suited to abstract ensemble models and simplify them. This is however not to be
mistaken with creating surrogates from the given models, as it represents the underlying
models exactly. By breaking down a RF in said rules, it is possible to represent the entire
RF in a matrix structure, by represeting columns as features and rows as rules, resulting in
cells as \textit{rules predicates}. This entire rethinking of the RF as a whole allows for
unique graphs and insights in the RFs' decision making. Thinking of the RF as a set of
comparable rules and evaluating those, yields a very deep understanding, both on a
case-based sample level, but also on a global level. While not supported in the original
paper, this would also allow for intricate comparisons of models on the same data set.
The biggest issue with this approach is, that while it does yield powerful insights, it can
be very difficult to convey these insights to a domain expert. While a data scientist can be
expected to wrap their head around the idea of disassembling a RF into a set of rules, this
idea is not intuitive for a domain expert who might already have trouble understanding how
the RF works in the first place.

% Look for RAFT paper in firefox bookmarks

\section{Summary}
While the main challenge of gaining access to the insights and details of the inner workings
of a RF might already have been solved they seem to be hidden behind a kind of complexity
layer. Powerful approaches for RF visualization exist, but there is a need to make them more
accessible for domain experts. There is a lot of value to be derived from the existing work
and they will surely be an influence to possible approaches aimed at domain experts.

\clearpage
\bibliographystyle{apacite}
\bibliography{ExposeReferences}
\end{document}