\documentclass[a4paper, 12pt]{article}

\usepackage[T1]{fontenc}
\usepackage{amsmath}
\usepackage{apacite}
\usepackage{enumitem}


\title{RF Research}
\author{Fabio Rougier}
\date{\today}

\begin{document}

\maketitle

\tableofcontents

\section{Master thesis from Eindhoven}
This \shortcite{kuznetsova2014random} is a master's thesis

\subsection{Remarks}
The author uses an entire chapter to describe in detail how a random forest works. 
It's definitely not a good thesis, judging by the wording. However, it might hold some value from its content\dots

\subsection{Points to note}
\begin{description}[font=$\bullet$~\normalfont]
    \item [Prototypes:]On p 11 the author mentions the idea of prototypes for every class and attribute
    \item [Different consumers: ]On p 14 the author emphasizes the different approach to ML, data and its 
        different consumers.
        While the ML expert is more focused on improving his model performance (and it's indicating variables.),
        the Analyst/Domain Expert is more concerned about the insights that the model yields (attributes, instances,
        noise,).
    \item [RAFT:]Random Forest Tool by Breimann is introduced as an existing tool for RF visualisation, authors
            conclusion is that tool is best with small amount and exclusively numerical attributes 
    \item [Small multiples:]Mentioned in her source n16 \cite{tufte1985visual}
    \item [Visualization considerations:] p20/21 lists concise considerations for the Visualization of the RF
            following the principle of small multiples
    \item [ReFINE:]Tool developed by the author, looks promising
\end{description}

\subsection{Summary}
Overall, a mediocre thesis, but all the more interesting and extensive software and approach. Note that it is from
2014 which is likely why the author uses java.

\section{Breimann Implementation Paper}
This \cite{livingston2005implementation} is a paper on a particular implementation on the RAFT software of Breimann.
\subsection{Remarks}
The paper seems to be concerned about the specifics of how a RF is created. Using RFs for problems where a very
rare occasion has to be trained often results in bad models, because the training set is already heavily scewed
towards the "regular" case and so the detection of edge cases seems to be a problem for such models.
They do use the RAFT software but the paper is likely not focusing on the specifics of RF Visualization.
Additionally the paper is from 2005, so it is quite outdated. They use Fortran and Java \dots

\subsection{Points to note}
\begin{description}[font=$\bullet$~\normalfont]
    \item[Weka:] Some kind of java program
    \item[Variable Importance:] The authors implemented variable importance into Weka
\end{description}

\subsection{Summary}
Very old paper and the specifics will unlikely be relevant today, however it might contain very important
references that I can use!

\section{Explainable Matrix paper}
This \cite{neto2020explainable} is a journal article presenting ExMatrix - a visualization method trying to convey
the configuration of a model in a matrix structure.

\subsection{Remarks}
This paper is from 2021 and therefore far more relevant than the others. It also specifically focuses on the
problematics of RFs.

\subsection{Points to note}
\begin{description}[font=$\bullet$~\normalfont]
    \item[Model interpretability:]The main problem statement is the lack of interpretability of models and their
            decisions despite being accurate. A 99\% accuracy does not convince anyone, if there is no
            explanation for the decision.
    \item [Global/Local approaches:]Global explains the entire model, trying to improve the trust in the models'
            decision making. Local explains the decision behind a single instance.
    \item[pre-/in-/post-model strategies:]For which stage of the ML process is the visualization helpful?
    \item[BaobabView:]This is a node-link explanation technique \cite{van2011baobabview}, down below. But node-link
            visualization has scalability issues.
    \item [RuleMatrix:]The technique has been used before \cite{ming2018rulematrix}.
    \item [Decision Paths:]Focus of the visualization is on decision paths, rather than nodes.
    \item [Surrogates:]Sometimes RFs are used as a \textit{surrogate} for a less interpretable model.
    \item [iForest:]Also closely related to ExMatrix \cite{zhao2018iforest}, summarizes decision paths.
\end{description}
Trying to explain the visualization specifics, because I think, this is an incredibly valuable paper for me:
Every path is translated into a rule vector, consisting of as many coordinates, as there are features along the 
path. Each rule consists of \textit{predicates} which represent single decisions on attribute values. 
For each rule vector there is a rule certainty (for each class, adding up to 100\%), which represents how
accurately the respective path classifies the instances that came along its path. The rule vector class is the
class that has the highest value in the rule certainty. Rule coverage is the percentage of instances of the
training data of class c for which the rule \textit{works}.
The 2nd graphic is the LE (Local Explanation) / UR (Used Rules) visualization and focuses on a single instance.
Each features decision boundaries are layed out, with a pointed line, representing the given instance.
There is an additional column \textit{cumulative voting} next to the rule certainty, which sums the certainties
up to the respective row, starting from the top.
The 3rd graphic is the LE / SC (smallest changes) visualization. It visualizes the smallest necessary rule change
to change the classification of the displayed sample.

\subsection{Summary}
Extremely relevant paper with lots of potential references and ideas. This could even be a visualization to build
my own visualization upon. The authors noted in their own conclusion, that especially the LE/SC visualization
leaves quite some room for improvement. The general idea of breaking down the RF into single rules is also
interesting. The code is on github and a forked version would be an option. However, the github was not updated
since last year April. I could consider contacting the authors if I pursue any further ideas along their approach.

\section{BaobabView}
This \cite{van2011baobabview} is a paper about the visualization of decision trees with the focus on giving domain
experts the ability to bring in their specific knowledge.
\subsection{Remarks}
Since this is mostly about decision trees, it might not be very applicable to RFs. It is also quite old already
(2011).
\subsection{Points to note}
\begin{description}[font=$\bullet$~\normalfont]
        \item[User requirements]Users of the visualization usually only want 3 different things: EDIT the tree (grow,
        prune or optimize) USE it (classification) or ANALYSE it (data exploration)
        \item [Node-Link diagram:] Visualization uses a Node-Link diagram
        \item [Streamgraphs:]Figure 4, might be interesting for Node visualization
        \item [Confusion Matrix:]Should be considered to inspect misclassifications
\end{description}
\subsection{Summary}
"Different activities require different visualizations"(in 6, Conclusion) is a simple, but important lesson from
this paper.

\section{Matrix Visualization}
This \cite{ming2018rulematrix} is an entire paper on the idea of a rule matrix. This is used in the aforementioned
"explainable Matrix" paper.
\subsection{Remarks}
There will likely be little that I can draw from this in regards to visualization of RFs, but it might still be a
valuable source for citation of basic concepts if I do go further down along the explainable matrix approach.
Since the paper was written in 2019 it is also one of the newer concepts.
\subsection{Points to note}
\begin{description}[font=$\bullet$~\normalfont]
        \item[Not specific to RFs:]The concept of turning a ML model into a rule matrix works with all kinds of ML
        models, not just RFs
\end{description}
\subsection{Summary}
Mostly included this paper for completeness. It will only become relevant if I dive deeper into the ExMatrix topic.

\section{iForest}
This \cite{zhao2018iforest} is a visualization approach mostly aiming for the interpretability of RFs.

\subsection{Remarks}
The version that I have seems to be the "unreleased" version, so be careful when citing this paper and look up the
links at the footer.
\subsection{Points to note}
\begin{description}[font=$\bullet$~\normalfont]
        \item[Poor interpretability:]Some domains would not use RFs simply because they are too difficult to
        interpret. Domains like healthcare or finance need the interpretability in order to be viable in day to day
        use.
        \item[3 different RF interpretation methods:] feature analysis, model reduction and case.based reasoning
        \item[Feature Analysis:]Calculation of feature importance is easy and effective (Either Mean Decrease 
        Accuracy ir Mean Decrease Impurity); However, MDI is specialised for tree-based models. Both can be calculated
        globally or for an individual feature.
        \item[Partial Dependence Plots(PDP):]Line chart where x-Axis is feature values and y-axis shows prediction
        probabilities. \cite{friedman2001greedy}, \cite{hastie2009elements}
        Usually good for trees, but not as good for forests. But since they included them anyways, there will
        probably follow a good reason for how and why they included them.
        \item[Model Reduction:] Mentions the idea of surrogates. Since they usually reduce too much and don't help
         with the actual understanding of the RF, they are not used here.
        \item[Case-based Reasoning:]Picking some kind of idealised sample in order to inspect the results of the RF
        based on the specific sample and user-configured permutations of that sample. This is connected to looking
        at distance measures between features and inspect correlations of distances and predictions.
        \item[Design Goals:]A section that I have seen in multiple papers now. Splitting up the different goals of
        the visualization and the necessary tasks to get there in G1,2,3\dots and T1,2,3\dots
        \item[Implementation:] This is one of the very few visualization where the implementation gets
        mentioned. Apparently they used Python (yay!). \textit{Scikitlearn} is used for the entire RF part and
        \textit{Flask} \cite{grinberg2018flask} for the Backend. The frontend is built using \textit{D3} \cite{d3js}.
\end{description}
\subsection{Summary}
Overall this visualization approach has better scalability than the matrix idea and the icicle representation from
the master's thesis. The implementation approach of designing the entire visualization as a web-application also
seems very promising.

\section{Current Bottom Line}
\subsection{Technical stuff}
The implementation from iForest seems very solid, but also a lot of work to set up initially. A web application
would be a modern approach and an excellent showcase for any kind of visualization.

\subsection{Target audience}
Focusing on a specific target audience seems to be useful as this entire ordeal is a constant decision about
trade-offs. Focusing on a target audience would reduce the amount of constant arbitrage between different kinds of
graphics significantly.

\subsection{Explicit goal}
Tied in with the target audience are the goals that the visualization wants to achieve. They will be helpful to
have an orientation during development but also give the thesis clearer path. Do I want DSs to precisely optimize
their RF model or do I want a doctor to understand why the model chose a specific class for a sample?

\subsection{Honest reflection on the necessity of the initial thesis statement}
A significant amount of high quality work has been done on the topic of visualizing RFs. Is it really necessary to
build something completely new from scratch or is there a smarter way to go about this?
Maybe it is possible to improve on one of the existing approaches. This would reduce development time significantly
(at least in theory, depending on how well documented the existing implementations are\dots) and enable more 
in-depth apporaches, that I would otherwise not be able to achieve if I was starting from nothing.
In order to do that, I would need to formulate a strong argument for why it is necessary to improve the existing
works and how this would be enough to warrant an entire thesis on it.

\subsection{Discussion with Aleksandar}
\begin{description}[font=$\bullet$~\normalfont]
        \item[Explainations:]How much explanation of RFs/Decision Trees concepts and algorithms is enough/too much?
        \item[Other ensemble methods:]Should I only focus on RFs?
        \item[Online vs. books:]Is it considered a problem that most of the resources are journal articles instead
        of books?
        \item[Research question:]Is it still to be formulated or was there a clear idea present?
        \item[Target audience:]Visualization needs will differ depending on who is supposed to use it.
        \item[Evaluation:]Is this an important part of the thesis or is a use case presentation sufficient?
        \item[Prototypes:]Relevant concept? (Idealised samples)
        \item[Scope:]Data preparation, Model creation, Data/Feature Analysis, Model performance\dots What to include?
        \item[\textit{Bad} Sources:]How to deal with sources that are obviously lackluster, but have important ideas?
        \item[Technical explanations:]Many of the works do not even mention implementation details. Why?
        \item[Breimann:]I read this name a lot in connection with basic concepts. Did he 'invent' RFs?
        \item[Build new or improve old:]Is it really necessary to build something new from scratch? How about
        improving or extending existing work? 
\end{description}

% This is where the references are created
% They "speak" with NotesReferences.bib
\clearpage
\bibliographystyle{apacite}
\bibliography{NotesReferences}
\end{document}