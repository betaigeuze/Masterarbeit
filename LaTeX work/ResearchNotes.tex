\documentclass[a4paper, 12pt]{article}

\usepackage[T1]{fontenc}
\usepackage{amsmath}
\usepackage{apacite}
\usepackage{enumitem}


\title{RF Research}
\author{Fabio Rougier}
\date{\today}

\begin{document}

\maketitle

\tableofcontents

\section{Master thesis from Eindhoven}
This \shortcite{kuznetsova2014random} is a master's thesis

\subsection{Remarks}
The author uses an entire chapter to describe in detail how a random forest works. 
It's definitely not a good thesis, judging by the wording. However, it might hold some value from its content\dots

\subsection{Points to note}
\begin{description}[font=$\bullet$~\normalfont]
    \item [Prototypes:]On p 11 the author mentions the idea of prototypes for every class and attribute
    \item [Different consumers: ]On p 14 the author emphasizes the different approach to ML, data and its 
        different consumers.
        While the ML expert is more focused on improving his model performance (and it's indicating variables.),
        the Analyst/Domain Expert is more concerned about the insights that the model yields (attributes, instances,
        noise,).
    \item [RAFT:]Random Forest Tool by Breimann is introduced as an existing tool for RF visualisation, authors
            conclusion is that tool is best with small amount and exclusively numerical attributes 
    \item [Small multiples:]Mentioned in her source n16 \cite{tufte1985visual}
    \item [Visualization considerations:] p20/21 lists concise considerations for the Visualization of the RF
            following the principle of small multiples
    \item [ReFINE:]Tool developed by the author, looks promising
\end{description}

\subsection{Summary}
Overall, a mediocre thesis, but all the more interesting and extensive software and approach. Note that it is from
2014 which is likely why the author uses java.

\section{Breimann Implementation Paper}
This \cite{livingston2005implementation} is a paper on a particular implementation on the RAFT software of Breimann.
\subsection{Remarks}
The paper seems to be concerned about the specifics of how a RF is created. Using RFs for problems where a very
rare occasion has to be trained often results in bad models, because the training set is already heavily scewed
towards the "regular" case and so the detection of edge cases seems to be a problem for such models.
They do use the RAFT software but the paper is likely not focusing on the specifics of RF Visualization.
Additionally the paper is from 2005, so it is quite outdated. They use Fortran and Java \dots

\subsection{Points to note}
\begin{description}[font=$\bullet$~\normalfont]
    \item[Weka:] Some kind of java program
    \item[Variable Importance:] The authors implemented variable importance into Weka
\end{description}

\subsection{Summary}
Very old paper and the specifics will unlikely be relevant today, however it might contain very important
references that I can use!

\section{Explainable Matrix paper}
This \cite{neto2020explainable} is a journal article presenting ExMatrix - a visualization method trying to convey
the configuration of a model in a matrix structure.

\subsection{Remarks}
This paper is from 2021 and therefore far more relevant than the others. It also specifically focuses on the
problematics of RFs.

\subsection{Points to note}
\begin{description}[font=$\bullet$~\normalfont]
    \item[Model interpretability:]The main problem statement is the lack of interpretability of models and their
            decisions despite being accurate. A 99\% accuracy does not convince anyone, if there is no
            explanation for the decision.
    \item [Global/Local approaches:]Global explains the entire model, trying to improve the trust in the models'
            decision making. Local explains the decision behind a single instance.
    \item[pre-/in-/post-model strategies:]For which stage of the ML process is the visualization helpful?
    \item[BaobabView:]This is a node-link explanation technique \cite{van2011baobabview}, down below. But node-link
            visualization has scalability issues.
    \item [RuleMatrix:]The technique has been used before \cite{ming2018rulematrix}.
    \item [Decision Paths:]Focus of the visualization is on decision paths, rather than nodes.
    \item [Surrogates:]Sometimes RFs are used as a \textit{surrogate} for a less interpretable model.
    \item [iForest:]Also closely related to ExMatrix \cite{zhao2018iforest}, summarizes decision paths.
\end{description}
Trying to explain the visualization specifics, because I think, this is an incredibly valuable paper for me:
Every path is translated into a rule vector, consisting of as many coordinates, as there are features along the 
path. Each rule consists of \textit{predicates} which represent single decisions on attribute values. 
For each rule vector there is a rule certainty (for each class, adding up to 100\%), which represents how
accurately the respective path classifies the instances that came along its path. The rule vector class is the
class that has the highest value in the rule certainty. Rule coverage is the percentage of instances of the
training data of class c for which the rule \textit{works}.
The 2nd graphic is the LE (Local Explanation) / UR (Used Rules) visualization and focuses on a single instance.
Each features decision boundaries are layed out, with a pointed line, representing the given instance.
There is an additional column \textit{cumulative voting} next to the rule certainty, which sums the certainties
up to the respective row, starting from the top.
The 3rd graphic is the LE / SC (smallest changes) visualization. It visualizes the smallest necessary rule change
to change the classification of the displayed sample.

\subsection{Summary}
Extremely relevant paper with lots of potential references and ideas. This could even be a visualization to build
my own visualization upon. The authors noted in their own conclusion, that especially the LE/SC visualization
leaves quite some room for improvement. The general idea of breaking down the RF into single rules is also
interesting. The code is on github and a forked version would be an option. However, the github was not updated
since last year April. I could consider contacting the authors if I pursue any further ideas along their approach.

\section{BaobabView}
This \cite{van2011baobabview} is a paper about the visualization of decision trees with the focus on giving domain
experts the ability to bring in their specific knowledge.
\subsection{Remarks}
Since this is mostly about decision trees, it might not be very applicable to RFs. It is also quite old already
(2011).
\subsection{Points to note}
\begin{description}[font=$\bullet$~\normalfont]
        \item[User requirements]Users of the visualization usually only want 3 different things: EDIT the tree (grow,
        prune or optimize) USE it (classification) or ANALYSE it (data exploration)
        \item [Node-Link diagram] Visualization uses a Node-Link diagram
        \item [Streamgraphs]Figure 4, might be interesting for Node visualization
        \item [Confusion Matrix]Should be considered to inspect misclassifications
\end{description}
\subsection{Summary}
"Different activities require different visualizations"(in 6, Conclusion) is a simple, but important lesson from
this paper.

\section{Matrix Visualization}
%AND THEN THIS:
\cite{ming2018rulematrix}

\subsection{Remarks}
\subsection{Points to note}
\subsection{Summary}

\section{iForest}
%AND THIS:
\cite{zhao2018iforest}

\subsection{Remarks}
\subsection{Points to note}
\subsection{Summary}

% This is where the references are created
% They "speak" with NotesReferences.bib
\clearpage
\bibliographystyle{apacite}
\bibliography{NotesReferences}

\end{document}