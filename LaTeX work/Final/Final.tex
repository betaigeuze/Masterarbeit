\documentclass[a4paper, 12pt]{article}
\usepackage[T1]{fontenc}
\usepackage{amsmath}
\usepackage{apacite}
\usepackage{enumitem}


\title{Master Thesis - Explainable Machine Learning - Visualization of Random Forests}
\author{Fabio Rougier}
\date{\today}

\begin{document}

\maketitle

\clearpage
\section{Summary}
DO SUMMARY AT THE END

\tableofcontents
\clearpage

\section{Introduction}
Random Forests (RF) are a powerful ensemble method with a low barrier of entry. Because of their
ease of use and performance they are used in many applications. However, they fall short
when it comes to transparency.
On the one hand RFs can offer a multitude of valuable insights into their decision making and the data they
process. On the other hand visualizing this is usually a challenge because of the inherent scale of a RF and the
aggregation of their decision making process. The goal of this work was to provide relative
beginners with a tool to explore a RF on a detailed level.

\subsection{Visualizing Decision Trees}
An obvious approach to visualizing a RF is inspecting the decision trees that constitute
the RF. One example for the visualization of a decision tree is \textit{BaobabView}
\cite{van2011baobabview}. As many other approaches visualizing decision trees, it utilizes
Node-Link Diagrams (NLDs) as its' main visualization. A confusion matrix yields additional
insights into the relations of the underlying features.
However this visualization does fall short when trees grow too large, as it becomes hard to
inspect every individual node and branch of the tree.
This is one of the problems that \textit{TaxonTree} tries to overcome
\cite{parr2003taxontree}. It uses a tree visualization approach that is scalable for large
trees by adding the possibility to zoom, browse and search the tree. While this does
help if a tree grows too large, it does not provide any insight on the general structure of
the tree as a whole.
\linebreak
Generalizing either of the approaches towards RFs is also non trivial. Both simply lack the
scalability in the desired dimension. It also leads into a dangerous territory of focusing too much on
the structure of individual trees. RFs - as all ensemble methods - arrive at their decisions
by combining the decisions of the individual trees. Inspecting and even understanding
individual trees, will only yield limited insights over the RF.

\subsection{Visualizing Random Forests}
To fully understand the structure and decision making of a RF, many aspects of the RF have to
be conveyed by the visualization. First of all typical indicators like a \textit{F1 score}
give an indication of the overall performance. Additionally there are some metrics specific
to RFs that should be included to get a deeper understanding of a particular instance of the
RF, like the \textit{mean impurity} of the individual trees or the \textit{out of bag error}.
With its' unique way of providing a distance measure for features,
a confusion matrix can also yield valuable insights to the relations of the features and how
the RF interprets them. As stated before, the visualization of RFs heavily relies on how it
overcomes the scalability issues of RFs.

\section{Related Work}

\subsection{ReFine}
This approach is rather old, but still worth mentioning, because it highlights the fact,
that visualizing RFs has been a challenge for some time now \cite{kuznetsova2014random}.
In this particular case, the technical implementation is of course out-dated, but
the ideas are still relevant. \textit{ReFine} uses a small multiples view of the trees to give the user
insights from different angles. The main visualization uses icicle plots, as the author
deems them most suitable due to their efficient use of space. While this does allow to
show a considerable number of trees in one view, it does not scale well enough for large
amounts of trees in a RF.
An important distinction raised by the author is that there are different users for RFs,
with very different requirements towards a RF visualization. While a machine learning
expert might be more focused on improving model performance, an analyst or domain expert
would be more concerned about the insights provided by the models.

\subsection{iForest}
The \textit{iForest} visualization is one of the most promising visualization approaches for RFs.
\cite{zhao2018iforest} It focuses on the interpretability of the RF and raises the concern
that many domains would not even consider using RFs because of their lack thereof.
The authors also approach the understanding of RFs by coming from two different angles:
\textit{Feature Analysis} and \textit{Case Based Reasoning}.
Both require different charts and give the user valuable insight into the dataset and
the RF. The elaborate, dashboard-like web application utilizes the benefits
of small multiples and has interconnected and interactive charts, each devoted to offer a
specific perspective. One especially unique part of the dashboard is the use of
\textit{Partial Dependence Plots} to display the RF's classification behavior in regard to
each feature. This is supported by a bar chart of the feature's distribution in order to
support this view with more context.
While the dashboard is a powerful tool, it can be overwhelming for users. It also
requires some in-depth knowledge about RFs in order to come to conclusions or even an
actionable instruction. \textit{iForest} is clearly aimed at supporting data scientists to understand
their own creations and not suited for a domain expert.

\subsection{ExMatrix}
The \textit{ExMatrix} follows an out-of-the-box-thinking approach by breaking up machine
learning models into a set of rules \cite{neto2020explainable} \cite{ming2018rulematrix}.
It is therefore even more flexible because it is not limited
to be used with RFs only, but could be applied to almost any kind of machine learning model.
It is best suited to abstract ensemble models and simplify them. This is however not to be
mistaken with creating surrogates from the given models, as it reflects the underlying
models exactly. By breaking down a RF in said rules, it is possible to display the entire
RF in a matrix structure. It does this by representing columns as features and rows as rules, resulting in
cells as so-called \textit{rules predicates}. This entire rethinking of the RF as a whole allows for
unique graphs and insights in the RFs' decision making. Thinking of the RF as a set of
comparable rules and evaluating those, yields a very deep understanding, both on a
case-based sample level, but also on a global level. While not supported in the original
paper, this would also allow for intricate comparisons of models on the same data set.
The biggest issue with this approach is, that while it does yield powerful insights, it can
be very difficult to convey these insights to a domain expert. While a data scientist can be
expected to wrap their head around the idea of disassembling a RF into a set of rules, this
idea is not intuitive for a domain expert who might already have trouble understanding how
the RF works in the first place.

\subsection{Summary}
While the main challenge of gaining access to the insights and details of the inner workings
of a RF might have been solved already they seem to be hidden behind a kind of complexity
layer. Powerful approaches for RF visualization exist, but there is a need to make them more
accessible for domain experts. Considering not only the audience, but also the intended
use of the visualization is paramount for the visualization to be useful. There is a lot
of value to be derived from the existing work and some of the ideas have influenced this
work.

\section{Methodology}
\subsection{Python}
The implementation of this work was done in \textit{Python 3.10.4} \cite{10.5555/1593511}, as it has
become one of the standard programming languages for data science and machine learning.
This allowed the usage of many commonly used libraries, like \textit{NumPy} \cite{harris2020array},
\textit{pandas} \cite{mckinney-proc-scipy-2010}, \textit{scikit-Learn} \cite{scikit-learn} and more.
Using Python also opens up the possibility to further extend this work in the future, while
the libraries and possibilities of Python keep improving and expanding.
Some of the libraries used in this work make use of \textit{Cython} \cite{behnel2011cython} to
speed up the execution of certain functions. This was particularly relevant for this work, as will
be elaborated in a later section.
We will go in depth on some of the libraries used in the following sections.

\subsection{pandas}
The \textit{pandas} library offers a powerful data structure called \textit{DataFrame}, which in
essence is just a table. Its' implementation is made so convenient and efficient though, that it
allows for both intuitive and performant data manipulation. \newline
In this work, \textit{pandas} is used as a backbone throughout the entire lifecycle of the
visualization.

\subsection{scikit-Learn}
With the \textit{scikit-Learn} library, machine learning is made easily accessible, even to
beginners, while still providing some more advanced configurations if necessary. \newline
For this work, both the \textit{RandomForestClassifier} and the \textit{DecisionTreeClassifier}
are a main part of the processes in the background.

\subsection{Streamlit}
\textit{Streamlit} provides users with an incredibly easy entry point to transferring Python
visualizations into the web browser. With a simple \textit{"st.write()"} command, any text,
visual or even {DataFrame} can be displayed in the browser. While it certainly has its' limits,
it is a great tool for a fast development of dashboards and visualizations. \newline
\textit{Streamlit} was used in this work to handle the web application and the user interface
that enables interactions with the visualizations and algorithms.

\subsection{Vega-Altair}
Speaking of visualizations, \textit{Vega-Altair} \cite{VanderPlas2018} is a great library for
creating interactive charts, with a great deal of flexibility for customization. It is built
on top of \textit{Vega-Lite} \cite{Satyanarayan2017}, which is a declarative grammar using
the JSON standard for building graphs, that are then rendered using the \textit{Vega-Lite}
compiler. \textit{Vega-Altair}, or \textit{Altair} for short, is especially unique in its'
way of constructing graph, by using \textit{encodings} for each visual element. This syntax
is very intuitive to use makes the library both powerful and easy to use. \newline
Every chart in this work is created using \textit{Altair}, which in turn allows the use of
\textit{Altair's} interactive capabilities like brushing and linking.

\subsection{NetworkX}
\textit{NetworkX} \cite{SciPyProceedings_11} is one of the most popular libraries for
manipulating graphs in Python. It contains many useful functions for graph manipulation.
\newline
Decision Trees are essentially binary trees, which can be represented as graphs. As such,
\textit{NetworkX} was used for some particular graph operations like the graph edit distance.
In order to transform the \textit{DecisionTreeClassifier} into a graph, \textit{PyGraphviz}
was used as an intermediate step.


\clearpage
\bibliographystyle{apacite}
\bibliography{FinalReferences}
\end{document}